% -*- TeX-master: "agda-hsp.tex" -*-
\documentclass[a4paper,UKenglish,cleveref,autoref,thm-restate]{lipics-v2021}
\usepackage{wjd}
% \usepackage{fontspec}
\usepackage{comment}
\usepackage{proof-dashed}
\usepackage{unixode}






\usepackage[dvipsnames]{xcolor}
\definecolor{britishracinggreen}{rgb}{0.0, 0.26, 0.15}
\hypersetup{
    bookmarks=true,         % show bookmarks bar?
    unicode=true,          % non-Latin characters in Acrobat’s bookmarks
    pdftoolbar=true,        % show Acrobat’s toolbar?
    pdfmenubar=true,        % show Acrobat’s menu?
    pdffitwindow=false,     % window fit to page when opened
    pdftitle={A Machine-checked Proof of Birkhoff's Variety Theorem},    % title
    pdfauthor={William DeMeo},     % author
    pdfsubject={formal verification},                 % subject of the document
    pdfcreator={pdflatex with hyperref},
    pdfproducer={},  % producer of the document
    pdfkeywords= {Agda} {Birkhoff’s HSP theorem} {equational logic}
    {Formalization of mathematics} {Martin-Löf Type Theory} {model theory}
    {formal verification}
    {universal algebra},
    pdfnewwindow=true,      % links in new window
    colorlinks=true,       % false: boxed links; true: colored links
    linkcolor=blue,          % color of internal links
    citecolor=black,        % color of links to bibliography
    filecolor=black,      % color of file links
    urlcolor=britishracinggreen           % color of external links
}

\usepackage[wjd,links]{agda}
\AgdaNoSpaceAroundCode{}

\newif\ifnonbold % comment this line to restore bold universe symbols

%%% HERE IS HOW WE CONTROL WHETHER LONG OR SHORT VERSION OF THE PAPER IS COMPILED
\newif\ifshort
\shorttrue % (to compile short version of the paper)
%\shortfalse % (to compile long version of the paper)
% USAGE: 
% \ifshort <stuff included in short version goes here> \else <stuff included in long version goes here> \fi
\newcommand\seeunabridged{see the unabridged version of this paper,~\cite{DeMeo:2021}, for the complete formalization\xspace}
\newcommand\seemedium{see~\cite{DeMeo:2021} for details\xspace}
\newcommand\seeshort{see~\cite{DeMeo:2021}\xspace}

% \usepackage[x11names, rgb]{xcolor}
\usepackage{tikz}
\usetikzlibrary{snakes,arrows,shapes}
\usepackage{amsmath}







%% VVVVVVVVVVVV BEGIN BETTER MARGIN NOTES  VVVVVVVVVVVVVVVVVVVVVVVVVVVVVVVVVVVVVVVV
%%
%% This section of macros provides for better margin notes.
%% For details, see:  http://tex.stackexchange.com/questions/9796/how-to-add-todo-notes
%%
\usepackage{xargs}    % for more than one optional parameter in a new
                      % command, use \newcommandx

\usepackage{datetime}
\usepackage[prependcaption,textsize=tiny]{todonotes}



%%  To make a simple, in-line note that appears within the text (not in the margin), use:
\newcommand{\wnote}[1]{\textcolor{Plum}{\noindent (WD: #1)}}
\newcommand{\jnote}[1]{\textcolor{blue}{\noindent (JC: #1)}}

% Otherwise, to make a (very brief) note that appears in a small box in the
% margin with a line pointing to nearby text, use 
\ddmmyydate
%% William's margin notes:  \wd{this is a note}
\newcommandx{\wjd}[2][1=]{%
  \todo[linecolor=purple,backgroundcolor=Plum!25,bordercolor=Plum,#1]%
       {{\bf wd} \today: #2}
}

%% Jacques' margin notes:  \jc{this is a note}
\newcommandx{\jc}[2][1=]{
  \todo[linecolor=blue,backgroundcolor=blue!25,bordercolor=blue,#1]%
       {{\bf jc} \today: #2}
}

\newcommandx{\thiswillnotshow}[2][1=]{\todo[disable,#1]{#2}}
%% ^^^^^^^^^ END BETTER MARGIN NOTES ^^^^^^^^^^^^^^^^^^^^^^^^^^^^^^^^^^^^^^^^^^^^^^^^
%%






\crefformat{footnote}{#2\footnotemark[#1]#3}

%%%%%%%%%%%%%% TITLE %%%%%%%%%%%%%%%%%%%%%%%%%%%%%%%%%%%%%%%%%%%%%%%%%%%%%%%%%%%%%%%%
% \ifshort
% \title{Birkhoff's Variety Theorem formalized in Agda}
% \else
\title{A Machine-checked Proof of Birkhoff's Variety Theorem in Martin-L\"of Type Theory}
\titlerunning{A Machine-checked Proof of Birkhoff's Theorem}
% \fi

%%%%%%%%%%%%%% AUTHOR %%%%%%%%%%%%%%%%%%%%%%%%%%%%%%%%%%%%%%%%%%%%%%%%%%%%%%%%%%%%%%%%
%\author{name}{affil}{email}{orcid}{funding}.
\author{William DeMeo}
       {New Jersey Institute of Technology}
       {williamdemeo@gmail.com}{https://orcid.org/0000-0003-1832-5690}{}
% partially supported by ERC Consolidator Grant No.~771005.}
\author{Jacques Carette}{McMaster University}{carette@mcmaster.ca}{https://orcid.org/0000-0001-8993-9804}{}
\authorrunning{DeMeo and Carette}

\copyrightfooter

% \pagestyle{fancy}
% \renewcommand{\sectionmark}[1]{\markboth{#1}{}}
% \fancyhf{}
% \fancyhead[ER]{\sffamily\bfseries \leftmark}
% \fancyhead[OL]{\sffamily\bfseries The agda-algebras library}
% \fancyhead[EL,OR]{\sffamily\bfseries \thepage}

\ccsdesc[500]{Theory of computation~Logic and verification}
\ccsdesc[300]{Computing methodologies~Representation of mathematical objects}
\ccsdesc[300]{Theory of computation~Type theory}
% \ccsdesc[300]{Theory of computation~Type structures}
% \ccsdesc[300]{Theory of computation~Constructive mathematics}

\keywords{Agda, constructive mathematics, dependent types, equational logic,
  formal verification, Martin-L\"of type theory, model theory, universal algebra}

\category{} %optional, e.g. invited paper

% \relatedversion{hosted on arXiv}
% \relatedversiondetails[linktext={http://arxiv.org/a/demeo\_w\_1}]{Part 2, Part 3}{http://arxiv.org/a/demeo_w_1}

% \supplement{~\\ \textit{Documentation}: \ualibdotorg}%
% \supplementdetails{Software}{https://gitlab.com/ualib/ualib.gitlab.io.git}

\nolinenumbers %uncomment to disable line numbering

\hideLIPIcs  %uncomment to remove references to LIPIcs series (logo, DOI, ...), e.g. when preparing a pre-final version to be uploaded to arXiv or another public repository

%Editor-only macros:: begin (do not touch as author)%%%%%%%%%%%%%%%%%%%%%%%%%%%%%%%%%%
\EventEditors{}
\EventNoEds{2}
\EventLongTitle{}
\EventShortTitle{}
\EventAcronym{}
\EventYear{2021}
\EventDate{26 October}
\EventLocation{}
\EventLogo{}
\SeriesVolume{0}
\ArticleNo{0}
%%%%%%%%%%%%%%%%%%%%%%%%%%%%%%%%%%%%%%%%%%%%%%%%%%%%%%

% \includeonly{4Algebras} %1Introduction,2Overture,3Relations,4Algebras,5Conclusion}

\begin{document}

\maketitle

%\newpage %%%%%%%%%%%%%%%%%%%%%%%%%%%%  TOC  %%%%%%%%%%%%%%%%%%%%%%

% \setcounter{tocdepth}{2}
% \tableofcontents

\begin{abstract}
The Agda Universal Algebra Library is a project aimed at formalizing the foundations of
universal algebra, equational logic and model theory in dependent type theory
using Agda. In this paper we draw from many components of the library to present
a self-contained, formal, constructive proof of Birkhoff's HSP theorem in
Martin-L\"of dependent type theory.
This achieves one of the project's initial goals: to demonstrate the expressive power of
inductive and dependent types for representing and reasoning about general
algebraic and relational structures by using them to formalize a significant theorem in the field.

\end{abstract}

\providecommand{\hypertarget}[2]{#2}
\providecommand\qv{\textit{q.v.}}
%% This is to make the typewriter font inserted by pandoc look more similar to
%% the Agda fonts used in the highlighted code.
\renewcommand\texttt[1]{\textsf{#1}}

% \input{../latex/Preface}
% \input{../latex/Overture}
% \input{../latex/Overture.Preliminaries}
% \input{../latex/Overture.Inverses}
% \input{../latex/Overture.Transformers}

% \let\paragraph\subsubsection
% \let\subsubsection\subsection
% \let\subsection\section
% \let\section\chapter
% \let\chapter\part

\input{Demos/HSP}
%\input{HSP} %% for arXiv version (where no subdirectories are allowed)

% \input{1Introduction}
% \input{2Overture}
% \input{3Relations}
% \input{4Algebras}
% \input{5Conclusion}
% \include{1Introduction}
% \include{2Overture}
% \include{3Relations}
% \include{4Algebras}
% \include{5Conclusion}


\paragraph*{Acknowledgments}
This work would not have been possible without the wonderful \agda language and
the \agdastdlib, developed and maintained by The Agda Team~\cite{agdastdlib}.
We thank the three anonymous referees for carefully reading the manuscript and
offering many excellent suggestions which resulted in a vast improvement in the
overall presentation.  One referee went above and beyond and provided us with a
simpler formalization of free algebras, in particular of \af{𝔽[~\ab{X}~]}.
This in turn led to simplifications of the proofs of the main theorem. We are
extremely grateful for this.
The first author was supported by the CoCoSym Project
under the ERC Consolidator Grant (ERC CoG), No.~771005.







%%%%%%%%%%%%%%%%%%%%%%%%%%%%%%%%%%%%%%%%%%%%%%%%%%%%%%%%%%%%%%%%%%%%%%%%%%%%%%%%%%%%%%%%%%
%%%%%%%%%%%%%%%%%%%%%%%%%%%%%%%%%%%%%%%%%%%%%%%%%%%%%%%%%%%%%%%%%%%%%%%%%%%%%%%%%%%%%%%%%%

%% Bibliography
\bibliographystyle{plainurl}

\bibliography{ualib_refs}

% \appendix
% \newpage
% \section{Dependency Graph}

% ~\hskip-1.5cm\includegraphics[scale=0.65]{ualib-graph-top.png}

% \newpage

% ~\hskip-2.5cm\includegraphics[scale=0.55]{ualib-graph-bot.png}


%% \section{Some Components of the Type Topology Library}
%% Here we collect some of the components from the \typetopology library that we used above but did not have space to discuss.  They are collected here for the reader's convenience and to keep the paper somewhat self-contained.

%% \input{aux/typetopology.tex}





%%%%%%%%%%%%%%%%%%%%%%%%%%%%%%%%%%%%%%%%%%%%%%%%%%%%%%%%%%%%%%%%%%%%%%%%%%%
\end{document} %%%%%%%%%%%%%%%%%%%%%%%%%%%%%%%%%%%%%%%%%%%%%%%%%%%%%%%%%%%%
%%%%%%%%%%%%%%%%%%%%%%%%%%%%%%%%%%%%%%%%%%%%%%%%%%%%%%%%%%%%%%%%%%%%%%%%%%%

